\documentclass[
	a4paper,
	11pt,
]{resume}

\usepackage{ebgaramond}
\usepackage{hyperref}

\name{Artin Ghasivand}

\address{\textbf{Homepage}: \href{https://artingh.com}{artingh.com}}
\address{\textbf{E-Mail}: ghasivand.artin@gmail.com}
\address{%
         \textbf{Haskell Gitlab}: \href{https://gitlab.haskell.org/Ei30metry}{gitlab.haskell.org/Ei30metry} \\
         \textbf{Github}: \href{https://github.com/Ei30metry}{github.com/Ei30metry}}

\begin{document}

\begin{rSection}{Education}
	\href{https://en.wikipedia.org/wiki/Alborz_High_School}{\textbf{Alborz High School}} \hfill \textit{January 2023} \\
	Diploma in Mathematics \& Physics \\
    Member of Biology Olympiad class \\
\end{rSection}

\begin{rSection}{Experience}

  \begin{rSubsection}{Cryptal Global}{June 2022 - December 2022}{Junior Developer}{Dubai, UAE}
  \item Automation bots in Python
  \item DevOps and Linux Administration
  \end{rSubsection}

\end{rSection}

\begin{rSection}{Volunteering}

  \begin{rSubsection}{Haskell Foundation}{Feb 2023 - Present}{Podcast quality inspector}{}
  \item Work with the editor and the hosts of the \href{https://haskell.foundation/podcast/}{Haskell interlude}
        podcast to make sure the episodes don't contain quality problems like noises or repetitive dialogues.
  \end{rSubsection}

\end{rSection}

\begin{rSection}{Projects}

  \href{https://gitlab.haskell.org/Ei30metry/haskell}{\textbf{Implementation of the GHC formalism paper (WIP)}}: Typechecker for Haskell \\
  A fresh Haskell-typechecker built on top of the formalism described in the
  \href{https://gitlab.haskell.org/Ei30metry/haskell/-/jobs/artifacts/wip/term/raw/haskell.pdf?job=build-pdf}
  {Typing of GHC Haskell, Part I (Early Draft). Artin Ghasivand, Simon Peyton Jones and Richard A. Eisenberg} paper, with key focuses of being:
  A one-to-one match to the inference rules,
  a robust way to test the underlying formalism,
  and a pedagogical tool for teaching about type-checkers and implementing them.

  \href{https://github.com/Ei30metry/blueprint}{\textbf{Blueprint (WIP)}}: Pretty-print outgoing call-hierarchies of Haskell functions \\
  The project uses the Glasgow Haskell Comiler as a \href{https://hackage.haskell.org/package/ghc}{library} to query the internal
  abstract syntax tree representation of Haskell functions and pretty-print it to the user.
  Although the project isn't finished yet, it helped me get familiar with GHC's internals, got accepted as one of Zurihac 2023's \href{https://zfoh.ch/zurihac2023/projects/}{projects}
  , and had a key role in helping me attend \href{https://haskell.foundation/events/2023-ghc-development-workshop.html}{GHC 2023 Contributors' workshop}.

  \href{https://github.com/Ei30metry/Hygeia}{\textbf{Hygeia (WIP)}}: A CLI program to keep track of your moods \\
  The project was initially started with the sole purpose of writing an actual program in Haskell that parsed
  a text file containing a special format with feeling entries, do some summarization, and report the pretty-printed
  output to the user.

  Key educational points of the project: A text parser written using \href{https://hackage.haskell.org/package/parsec}{parsec},
  a command line argument parser using \href{https://hackage.haskell.org/package/optparse-applicative}{optpars-applicative},
  a pretty-printer using \href{https://hackage.haskell.org/package/prettyprinter}{prettyprinter},
  and managed effects using the \href{https://hackage.haskell.org/package/mtl}{monad transformers} library.
  The older versions also used the \href{https://hackage.haskell.org/package/singletons}{singletons} library, which was later removed to simplify the codebase.
\end{rSection}

\begin{rSection}{Talks}
  A good programming language is a \emph{Functional} one, University of Tehran. (\href{https://github.com/Ei30metry/fp-haskell-talk-UT/blob/main/slides.pdf}{slides}, \href{https://www.youtube.com/watch?v=DbJGux_-aP0}{video})
\end{rSection}

\begin{rSection}{Technical Strengths}

	\begin{tabular}{@{} >{\bfseries}l @{\hspace{6ex}} l @{}}
      Programming Languages & Haskell, Agda, Emacs Lisp, Common Lisp, Python, C \\
      Other Languages & \LaTeX, ott \\
      Tools & Git, Nix, GNU Make, Docker \\
	\end{tabular}

\end{rSection}

\end{document}
